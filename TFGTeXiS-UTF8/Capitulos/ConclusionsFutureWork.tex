\chapter*{Conclusions and Future Work}
\label{cap:conclusions}
\addcontentsline{toc}{chapter}{Conclusions and Future Work}
\subsection{Conclusions}

First of all, we can affirm that, through the help of artificial intelligence, personalized photographic material can be provided that can be used to collaborate with occupational therapies, so that a patient has the possibility of evoking significant moments, which they would not otherwise have. the ability to visualize.\\

With respect to image generation, the objective of training an artificial intelligence model to convert text to image has been achieved, including the person considered. This fact means that, in reminiscence therapy, if we are given a considerable number of images in which a specific person appears (from ten), we can work based on a trained model that recognizes that person, and generate the desired images from it.\\

Additionally, the model that has been obtained can serve as a basis for carrying out new training, which fits perfectly with our objectives, since for a certain patient, we can develop a large model that includes the desired number of people. . However, Stable Diffusion and its derived models have a limitation, since the same image is difficult to generate with multiple elements. We have achieved acceptable results on some occasions with a considerable number of steps, but it requires a lot of patience and we cannot guarantee absolute success, as we can in a generation with a main element.\\

Another important aspect that we must take into account and that we have reached the conclusion after all the research, is that to make our model work, like any other image-generative artificial intelligence, a large capacity of graphic memory is needed. . With our team's GPU, the generation will always last a few minutes, and the more quality desired, the longer the time will increase. This is a fact that all people who provide the service of generating personalized photographs must know.\\

Regarding the training of the models, we can conclude that the number of steps with which it is performed correctly is 2400 for a training of 10 images, since multiple tests have been carried out with more steps, in which overtraining is detected, and with fewer steps, in which a lack of training is detected.\\

Based on the results obtained, we can affirm that the training that maximizes the quality of the images is that of Dreambooth, with a quality higher than that achieved with LoRA. With this last model, it was much more difficult to obtain personalized images from the trained element, since given its characteristics, it focused on displaying only the element, ignoring the complements that were requested in the prompt. Through Dreambooth, the images showed superior quality and everything that was requested in the description of the photograph was reflected. With both training models, we can conclude that all types of elements can be trained, whether people, animals or places, since correct results have been obtained with all of them.\\

Regarding the application, we have managed to create a program through which personalized photographs can be generated from our trained models. This produces very good results and also allows the visualization of a life book based on a story created by us. The idea is that a user can have a program with which they can generate personalized images and develop a very attractive life book in a simple way.\\

We can conclude that we have met the objectives we had previously at the beginning of this project, since we have built a program through which a user could obtain personalized images with a reasonable quality in an adequate time. There has been a constant evolution in recent months in which we went from not being able to generate images locally, to being able to generate them with trained models, and then being able to display those photographs in an application developed by us.\\

\subsection{Future work}

After carrying out this project, we would like people in charge of assisting patients with memory loss to be able to carry out research based on our model, with the aim of psychologically studying to what extent the reduction of stress is being promoted, is being enhanced. the patient's cognitive capacity and people's satisfaction is being achieved. This is the greatest objective we have with this artificial intelligence, to achieve personal well-being, and the scope of this work does not allow us to verify if we have really obtained great results in the social aspect.\\

Additionally, it would be convenient that if this project were used by therapists, they should have a computer with a large graphics card, so that as little time as possible would be invested in generating the images. After having used this same model in the cloud, where they use servers with high-power GPUs, obtaining results was instantaneous, which is why we highlight a large margin for improvement that, if reduced, will greatly speed up the therapy. Furthermore, it should be noted that in our model, we have used version 1.5 of Stable Diffusion because it is the only one that could work locally with our team's graphics card. With an improvement in this aspect, the XL version could be worked on and trained, which provides great quality to the images, and even better results would be achieved by minimizing the waiting time. In addition, Stable Diffusion has recently released a new version 3 that ensures speed, efficiency and quality in the images generated without needing to use as much space as the XL version, so it would be a very attractive option to work with. You could even take the work of life books one step further and generate small video clips that better represent the latest developments that large Artificial Intelligence companies such as OpenAI through Sora and Stable Diffusion are including. via Stable Diffusion Video. They are technologies that produce impressive results but that are not currently available to users but we hope that they will be in the not too distant future. \\

Regarding the training of images with Stable Diffusion, as proposals for improvement we would like to find an optimal way to include several tokens that represent trained elements in the same image and that it is not necessary to resort to the inpainting technique. Methods could be tested such as training more images of both elements in the same training stage and that they can be included under the same token, although the behavior that the model could have in terms of results is uncertain, multilayers are an interesting concept in terms of generation of images for life books. \\

Based on having made an application that tests the trained models and displays a life book with personalized images, we would like to implement a server so that performance depends exclusively on it and the internet connection, and not on the computer. of each user and their graphics card. In this way, strong utilization requirements would be eliminated. Furthermore, it would be convenient to continue developing this program, allowing users to register and manage the life book of reminiscence therapy patients. To carry out this proposal, the project would have to be connected to a database that contains the user registry, and within each of them, a table with the patient's life book, which should have as attributes, mainly, an image, a title and a description. We believe that it can be an attractive format for therapy, where users can very easily personalize a life book, with real images, or those created by artificial intelligence, in the event that there are no graphic documents of a relevant fact, meaningful and emotional for the patient's life.

