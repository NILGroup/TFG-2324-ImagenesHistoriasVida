\chapter{Introducción}
\label{cap:introduccion}

\chapterquote{Todo se hunde en la niebla del olvido pero cuando la niebla se despeja, el olvido está lleno de memoria}{Mario Benedetti}

Nuestra vida se compone desde los primeros instantes en los tomamos nuestro primer aliento hasta que respiramos el último. Sin embargo, a la hora de relatar nuestra vida, esta historia estaría compuesta solamente de recuerdos. Desde nuestro primer recuerdo, que normalmente está comprendido entre los 2 y 4 años, debido a que la formación de nuevas neuronas impide que la corteza almacene recuerdos hasta esa edad, hasta el último, el cual puede darse el caso de ser distorsionado o borrado completamente debido al deterioro de nuestro cerebro y la imposibilidad de nuestras neuronas a realizar las conexiones posibles para ello. Este último caso es una enfermedad degenerativa comúnmente conocida como Alzheimer, que actualmente afecta a más de 55 millones de personas en todo el mundo. Sin embargo, no son sólo estas personas quiénes la sufren, sino todos sus allegados también. \\


Para ayudar a combatir esta enfermedad existen métodos farmacológicos y no farmacológicos, en este proyecto nos centraremos en el segundo grupo, más concretamente, en la terapia de reminiscencia. En cuanto a los métodos que existen actualmente para realizar esta terapia se ha comprobado que la ayuda de apoyo visual es mucho más efectiva a entrevistas con muchas preguntas seguidas, ya que esta última puede provocar que el paciente se vea agobiado y abrumado. Es por este motivo, entre muchos otros, que nuestro proyecto está orientado a ser una herramienta de apoyo visual a la narración de libros de vida. Para ello haremos uso de Inteligencias artificiales que convierten entradas de texto a imágenes y de esta manera, podremos transformar recuerdos narrados por el paciente y convertirlos en imágenes. Lo interesante es poder dar la posibilidad de hacer estas imágenes más personales de forma que se puedan añadir una serie de imágenes propias,de familiares o de eventos importantes de la vida del paciente, para entrenar al modelo y que el resultado final sea único, especial y de gran ayuda para la persona que sufre de esta enfermedad. 

\section{Motivos}

El trabajo nace con base al proyecto CANTOR (Composición Automática de Narrativas personales como apoyo a Terapia Ocupacional basada en Reminiscencia, Plan Nacional de Investigación), una iniciativa planteada por investigadores de un conjunto de universidades españolas, entre ellas la Universidad Complutense de Madrid.

La principal motivación de la creación de este proyecto es brindar la posibilidad a los pacientes de demencia causada por el Alzheimer a construir un libro compuesto por sus recuerdos más felices, brindándole la oportunidad de rememorarlos. Se ha demostrado que cuando los pacientes hablan sobre determinadas épocas de su vida y momentos vividos en el pasado, se genera un impacto positivo en su persona consiguiendo que aumente su confianza y sentido de la vida.\\

El proyecto CANTOR está pensado para elaborarse en dos etapas: 
La primera sería la parte técnica en la que se elabora una herramienta que, mediante Inteligencia Artificial, para automatizar la construcción del libro de vida. 

La segunda etapa consiste en llevar la herramienta a los terapeutas y personas que asisten a los afectados para comprobar la funcionalidad y efectividad de la misma. 

Nosotros nos centraremos en abordar la primera etapa del proyecto, asegurándonos de implementar una herramienta capaz de generar imágenes a través de una inteligencia artificial especializada en ello, de manera que asegure la calidad máxima posible en los resultados y procurando una semejanza prácticamente total a la realidad.\\


\section{Objetivos}

- Utilizar la inteligencia artificial generativa para crear imágenes personales que ayuden a los pacientes a evocar momentos y experiencias emocionales e integrarlos en el presente.

- Generar fotografías que respalden historias de los pacientes.

- Brindar material de apoyo para la terapia de reminiscencia.

- Elaborar un programa mediante el cual, un ayudante pueda incorporar imágenes propias al modelo.

Por ejemplo: un paciente recuerda cuando vio el mar por primera vez, pero no conoce detalles suficientes como para tener integrada una historia que contar en la mente, ni tomó ninguna fotografía en aquel entonces. El modelo puede generar una foto del paciente en el mar, y el hecho de evocar ese recuerdo, le provoca bienestar y felicidad.

De esta manera, podemos aportar un material muy valioso para la terapia de reminiscencia, ya que se necesita material visual que permita crear una conexión con la vida del paciente, y este material en ocasiones puede ser muy limitado.

\section{Plan de trabajo}

Una vez definidos los objetivos, se debe establecer un método para tratar de llegar a los resultados esperados. En primer lugar, se debe realizar una amplia investigación acerca de los tipos de inteligencia artificial que existen y cuál de todas es la que mejor se adapta a nuestro objeto de estudio. Pero antes de profundizar en las diferentes técnicas y modelos, debemos ser conscientes del motivo por el que realizamos este trabajo, es decir, quién es el destinatario y qué espera del producto final. Por tanto, necesitamos indagar en el foco del problema y saber cómo la inteligencia artificial puede ayudar a resolverlo, o bien a mitigarlo. 

Sabiendo que para llegar a los resultados deseados necesitamos una inteligencia artificial generativa de imágenes a partir de texto, necesitamos conocer cuáles son las mejores que están a nuestra disposición, y si las podemos utilizar y trabajar sobre ellas. Por tanto, una parte importante de nuestro proyecto consiste en realizar pruebas de cada una de las inteligencias artificiales generativas y valorar cuál genera imágenes de mayor calidad y cuál la genera en un tiempo aceptable. Es necesario probarlas todas y cada una de las que estén disponibles, y saber cuál va a ser nuestro entorno de ejecución.

Una vez se haya decidido cuál va a ser el o los modelos que elijamos para desarrollar nuestro proyecto, llegará el momento de saber cómo vamos a desplegar las diferentes tecnologías y qué vamos a añadir para que sea algo útil y completamente novedoso para nuestros destinatarios. La idea es crear un programa, que contenga el modelo de inteligencia artificial elegido y una interfaz sencilla y eficaz para utilizarla en las terapias ocupacionales. Además, estos modelos deberán tener la posibilidad de ser personalizables, de manera que se puedan crear imágenes con los elementos o personas que se deseen. Para ello, se investigará sobre los distintos modos de entrenamiento, y realizando múltiples pruebas y analizando los diferentes resultados, se razonará cuál será el entrenamiento óptimo para nuestro proyecto.

Cuando ya disponemos de un modelo de inteligencia artificial generativa de imágenes que otorgue buenos resultados, simularemos casos de uso que ejemplifiquen cómo los usuarios pueden tener una experiencia plenamente satisfactoria. A raíz de esto, podremos obtener una serie de conclusiones y anotar si se han cumplido las expectativas y si en un futuro los pacientes pueden ver su calidad de vida incrementada gracias a nuestra iniciativa y a nuestro trabajo.


