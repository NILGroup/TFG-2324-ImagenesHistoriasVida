\chapter{Conclusiones y Trabajo Futuro}
\label{cap:conclusiones}

En primer lugar, podemos afirmar que, mediante la ayuda de la inteligencia artificial, se puede brindar un material fotográfico personalizado que sirva para colaborar con terapias ocupacionales, de manera que un paciente tenga la posibilidad de evocar momentos significativos, que de otra manera no tendría la capacidad de visualizar.\\

Con respecto a la generación de imágenes, se ha conseguido el objetivo de entrenar un modelo de inteligencia artificial de convertir texto a imagen, incluyendo a la persona que se considere. Este hecho significa que, en una terapia de reminiscencia, si se nos otorga un número considerable de imágenes en las que aparezca una persona concreta (a partir de diez), se puede trabajar en base a un modelo entrenado que reconozca a esa persona.\\
 
Adicionalmente, el modelo que se ha obtenido, puede servir de base para realizar un nuevo entrenamiento, lo cuál encaja a la perfección con nuestros objetivos, puesto que para un determinado paciente, podemos desarrollar un gran modelo que incluya el número de personas que se desee. \\

Otro aspecto importante que debemos tener en cuenta y al que hemos llegado a la conclusión después de toda la investigación, es que para hacer funcionar nuestro modelo, al igual que cualquier otro de inteligencia artificial generativa de imágenes, se necesita una gran capacidad de memoria gráfica. Con la GPU de nuestro equipo, la generación siempre va a abarcar unos minutos, y cuanta más calidad se desee, más tiempo se incrementará. Esto es un hecho que todas las personas que se presten al servicio de generar las fotografías personalizadas deben conocer.\\

Respecto al entrenamiento de los modelos, podemos concluir que el número de pasos pasos con el que se realiza correctamente es 2400 para un entrenamiento de 10 imágenes, puesto que se han realizado múltiples pruebas con más pasos, en los que se detecta sobreentrenamiento, y con menos pasos, en los que se detecta falta de entrenamiento.\\

En base a los resultados obtenidos, podemos afirmar que el entrenamiento que maximiza la calidad de las imágenes es el de Dreambooth.\\

En el caso de realizarse la aplicación, se incluyen más conclusiones\\

Propuestas de mejora\\

Tras la realización de este proyecto, nos gustaría que personas encargadas de asistir a pacientes con pérdida de memoria, pudiesen realizar una investigación partiendo de nuestro modelo, con el objetivo de estudiar psicológicamente en qué medida se está favoreciendo la reducción del estrés, se está potenciando la capacidad cognitiva del paciente y se está logrando la satisfacción de las personas. Este es el mayor objetivo que tenemos con esta inteligencia artificial, lograr el bienestar personal, y el alcance de este trabajo no nos permite comprobar si realmente hemos obtenido grandes resultados en el aspecto social.
Adicionalmente, sería conveniente que si este proyecto fuese utilizado por los terapeutas, deberían tener un equipo con una gran tarjeta gráfica, para que se invirtiera así el menor tiempo posible en la generación de las imágenes. Tras haber utilizado este mismo modelo en la nube, donde utilizan servidores con GPUs de gran potencia, la obtención de resultados era instantánea, por lo que destacamos un gran margen de mejora que en caso de reducirse, agilizará en gran medida la terapia. Además, cabe resaltar que en nuestro modelo, hemos empleado la versión 1.5 de Stable Diffusion porque es la única que podía funcionar en local con la tarjeta gráfica de nuestro equipo. Con una mejora en este aspecto, se podría trabajar y entrenar la versión XL, que aporta una gran calidad a las imágenes, y se lograrían unos resultados aún mejores minimizando el tiempo de espera.\\






