\chapter{Conclusiones y Trabajo Futuro}
\label{cap:conclusiones}

En primer lugar, podemos afirmar que, mediante la ayuda de la inteligencia artificial, se puede brindar un material fotográfico personalizado que sirva para colaborar con terapias ocupacionales, de manera que un paciente tenga la posibilidad de evocar momentos significativos, que de otra manera no tendría la capacidad de visualizar.\\

Con respecto a la generación de imágenes, se ha conseguido el objetivo de entrenar un modelo de inteligencia artificial de convertir texto a imagen, incluyendo a la persona que se considere. Este hecho significa que, en una terapia de reminiscencia, si se nos otorga un número considerable de imágenes en las que aparezca una persona concreta (a partir de diez), se puede trabajar en base a un modelo entrenado que reconozca a esa persona, y generar a partir de él las imágenes deseadas.\\

Adicionalmente, el modelo que se ha obtenido, puede servir de base para realizar un nuevo entrenamiento, lo cuál encaja a la perfección con nuestros objetivos, puesto que para un determinado paciente, podemos desarrollar un gran modelo que incluya el número de personas que se desee. No obstante, Stable Diffusion y sus modelos derivados, tienen una limitación, dado que una misma imagen es difícil de generar con múltiples elementos. Hemos logrado resultados aceptables en algunas ocasiones con un número considerable de pasos, pero requiere mucha paciencia y no podemos garantizar el éxito absoluto, como sí podemos hacerlo en una generación con un elemento principal.\\

Otro aspecto importante que debemos tener en cuenta y al que hemos llegado a la conclusión después de toda la investigación, es que para hacer funcionar nuestro modelo, al igual que cualquier otro de inteligencia artificial generativa de imágenes, se necesita una gran capacidad de memoria gráfica. Con la GPU de nuestro equipo, la generación siempre va a abarcar unos minutos, y cuanta más calidad se desee, más tiempo se incrementará. Esto es un hecho que todas las personas que se presten al servicio de generar las fotografías personalizadas deben conocer.\\

Respecto al entrenamiento de los modelos, podemos concluir que el número de pasos pasos con el que se realiza correctamente es 2400 para un entrenamiento de 10 imágenes, puesto que se han realizado múltiples pruebas con más pasos, en los que se detecta sobreentrenamiento, y con menos pasos, en los que se detecta falta de entrenamiento.\\

En base a los resultados obtenidos, podemos afirmar que el entrenamiento que maximiza la calidad de las imágenes es el de Dreambooth, con una calidad superior a la conseguida con LoRA. Con este último modelo, era mucho más difícil obtener imágenes personalizadas a partir del elemento entrenado, puesto que dadas sus características, se centraba en mostrar únicamente el elemento, ignorando los complementos que se pedían en el prompt. A través de Dreambooth, las imágenes mostraban una calidad superior y sí que se veía reflejado todo lo que se pedía en la descripción de la fotografía. Con ambos modelos de entrenamiento, podemos concluir que se puede entrenar todo tipo de elementos, ya sean personas, animales o lugares, ya que con todos ellos se han obtenido resultados correctos.\\

Con respecto a la aplicación, hemos conseguido realizar un programa, mediante el cual, se puedan generar fotografías personalizadas a partir de nuestros modelos entrenados. Esto produce muy buenos resultados y, además, permite la visualización de un libro de vida basado en una historia creada por nosotros. La idea es que un usuario pueda contar con un programa con el que pueda generar imágenes personalizadas y desarrollar un libro de vida muy atractivo de manera sencilla.\\

Podemos concluir con que hemos cumplido con los objetivos que teníamos previamente al comienzo este proyecto, puesto que hemos construido un programa mediante el cual un usuario podría obtener imágenes personalizadas con una calidad razonable en un tiempo adecuado. Ha habido una evolución constante en los últimos meses en los que se pasó de no poder generar imágenes localmente, a poder generarlas con modelos entrenados, para después poder mostrar esas fotografías en una aplicación desarrollada por nosotros.


