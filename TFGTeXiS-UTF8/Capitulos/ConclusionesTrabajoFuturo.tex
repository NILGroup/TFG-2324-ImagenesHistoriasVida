\chapter{Conclusiones y Trabajo Futuro}
\label{cap:conclusiones}

\section{Conclusiones}

En primer lugar, podemos afirmar que, mediante la ayuda de la inteligencia artificial, se puede brindar un material fotográfico personalizado que sirva para colaborar con terapias ocupacionales, de manera que un paciente tenga la posibilidad de evocar momentos significativos, que de otra manera no tendría la capacidad de visualizar.\\

Con respecto a la generación de imágenes, se ha conseguido el objetivo de entrenar un modelo de inteligencia artificial de convertir texto a imagen, incluyendo a la persona que se considere. Este hecho significa que, en una terapia de reminiscencia, si se nos otorga un número considerable de imágenes en las que aparezca una persona concreta (a partir de diez), se puede trabajar en base a un modelo entrenado que reconozca a esa persona, y generar a partir de él las imágenes deseadas.\\

Adicionalmente, el modelo que se ha obtenido, puede servir de base para realizar un nuevo entrenamiento, lo cuál encaja a la perfección con nuestros objetivos, puesto que para un determinado paciente, podemos desarrollar un gran modelo que incluya el número de personas que se desee. No obstante, Stable Diffusion y sus modelos derivados, tienen una limitación, dado que una misma imagen es difícil de generar con múltiples elementos. Hemos logrado resultados aceptables en algunas ocasiones con un número considerable de pasos, pero requiere mucha paciencia y no podemos garantizar el éxito absoluto, como sí podemos hacerlo en una generación con un elemento principal.\\

Otro aspecto importante que debemos tener en cuenta y al que hemos llegado a la conclusión después de toda la investigación, es que para hacer funcionar nuestro modelo, al igual que cualquier otro de inteligencia artificial generativa de imágenes, se necesita una gran capacidad de memoria gráfica. Con la GPU de nuestro equipo, la generación siempre va a abarcar unos minutos, y cuanta más calidad se desee, más tiempo se incrementará. Esto es un hecho que todas las personas que se presten al servicio de generar las fotografías personalizadas deben conocer.\\

Respecto al entrenamiento de los modelos, podemos concluir que el número de pasos pasos con el que se realiza correctamente es 2400 para un entrenamiento de 10 imágenes, puesto que se han realizado múltiples pruebas con más pasos, en los que se detecta sobreentrenamiento, y con menos pasos, en los que se detecta falta de entrenamiento.\\

En base a los resultados obtenidos, podemos afirmar que el entrenamiento que maximiza la calidad de las imágenes es el de Dreambooth, con una calidad superior a la conseguida con LoRA. Con este último modelo, era mucho más difícil obtener imágenes personalizadas a partir del elemento entrenado, puesto que dadas sus características, se centraba en mostrar únicamente el elemento, ignorando los complementos que se pedían en el prompt. A través de Dreambooth, las imágenes mostraban una calidad superior y sí que se veía reflejado todo lo que se pedía en la descripción de la fotografía. Con ambos modelos de entrenamiento, podemos concluir que se puede entrenar todo tipo de elementos, ya sean personas, animales o lugares, ya que con todos ellos se han obtenido resultados correctos.\\

Con respecto a la aplicación, hemos conseguido realizar un programa, mediante el cual, se puedan generar fotografías personalizadas a partir de nuestros modelos entrenados. Esto produce muy buenos resultados y, además, permite la visualización de un libro de vida basado en una historia creada por nosotros. La idea es que un usuario pueda contar con un programa con el que pueda generar imágenes personalizadas y desarrollar un libro de vida muy atractivo de manera sencilla.\\

Podemos concluir con que hemos cumplido con los objetivos que teníamos previamente al comienzo este proyecto, puesto que hemos construido un programa mediante el cual un usuario podría obtener imágenes personalizadas con una calidad razonable en un tiempo adecuado. Ha habido una evolución constante en los últimos meses en los que se pasó de no poder generar imágenes localmente, a poder generarlas con modelos entrenados, para después poder mostrar esas fotografías en una aplicación desarrollada por nosotros.\\

\section{Propuestas de mejora}

Tras la realización de este proyecto, nos gustaría que personas encargadas de asistir a pacientes con pérdida de memoria, pudiesen realizar una investigación partiendo de nuestro modelo, con el objetivo de estudiar psicológicamente en qué medida se está favoreciendo la reducción del estrés, se está potenciando la capacidad cognitiva del paciente y se está logrando la satisfacción de las personas. Este es el mayor objetivo que tenemos con esta inteligencia artificial, lograr el bienestar personal, y el alcance de este trabajo no nos permite comprobar si realmente hemos obtenido grandes resultados en el aspecto social.\\

Adicionalmente, sería conveniente que si este proyecto fuese utilizado por los terapeutas, deberían tener un equipo con una gran tarjeta gráfica, para que se invirtiera así el menor tiempo posible en la generación de las imágenes. Tras haber utilizado este mismo modelo en la nube, donde utilizan servidores con GPUs de gran potencia, la obtención de resultados era instantánea, por lo que destacamos un gran margen de mejora que en caso de reducirse, agilizará en gran medida la terapia. Además, cabe resaltar que en nuestro modelo, hemos empleado la versión 1.5 de Stable Diffusion porque es la única que podía funcionar en local con la tarjeta gráfica de nuestro equipo. Con una mejora en este aspecto, se podría trabajar y entrenar la versión XL, que aporta una gran calidad a las imágenes, y se lograrían unos resultados aún mejores minimizando el tiempo de espera.  Además, Stable Diffusion recientemente ha lanzado una nueva versión 3 que asegura rapidez, eficiencia y calidad en las imágenes generadas sin necesidad de utilizar tanto espacio como en la versión XL, por lo que sería una opción muy atractiva con la que trabajar. Incluso, podría llevarse el trabajo de los libros de vida un paso más allá y generar pequeños clips de vídeo que representen de mejor manera a través de las últimas novedades que están incluyendo las grandes compañías de Inteligencia Artificial como OpenAI a través de Sora y Stable Diffusion a través de Stable Diffusion Video. Son tecnologías que producen resultados impresionantes pero que ahora mismo no están al alcance de los usuarios pero que esperamos que en un futuro no muy lejano lo estén. \\

En cuanto al entrenamiento de imágenes con Stable Diffusion, como propuestas de mejora nos gustaría encontrar una forma óptima de incluir varios \textit{tokens} que representan elementos entrenados en una misma imagen y que no sea necesario recurrir a la técnica de \textit{inpainting}. Podrían probarse métodos como entrenar más imágenes de ambos elementos en una misma etapa de entrenamiento y que se puedan englobar bajo el mismo \textit{token}, aunque el comportamiento que podría tener el modelo en cuanto a resultados es incierto, las multicapas son un concepto interesante en cuanto a generación de imágenes para libros de vida. \\


En base a haber realizado una aplicación que ponga a prueba los modelos entrenados y que muestre un libro de vida con imágenes personalizadas, nos gustaría implementar un servidor de manera que el rendimiento dependa exclusivamente de él y de la conexión a internet, y no del equipo de cada usuario y su tarjeta gráfica. De esta manera, se eliminarían los requisitos fuertes de utilización. Además, sería conveniente seguir desarrollando este programa, permitiendo que los usuarios se registren y sean ellos quienes administren y creen el libro de vida de los pacientes de la terapia de reminiscencia, de modo que puedan ordenar las imágenes generadas cronológicamente y bajo una descripción del recuerdo. Para llevar a cabo esta propuesta, se tendría que conectar el proyecto a una base de datos, que contenga el registro de usuarios, y dentro de cada uno de ellos, una tabla con el libro de vida del paciente, que debería tener como atributos, principalmente, una imagen, un título y una descripción. Creemos que puede ser un formato atractivo para la terapia, donde los usuarios pueden personalizar de manera muy sencilla un libro de vida, con imágenes reales, o bien creadas por inteligencia artificial, en el caso de que no existan documentos gráficos de un hecho relevante, significativo y emocional para la vida del paciente.

Por otro lado, y en cuanto a usabilidad de la interfaz, se podría añadir un sistema de reconocimiento de personas y otros elementos en las que el propio usuario pueda elegir con qué elemento de los que ya ha entrenado previamente, desea generar una nueva imagen. E incluso, que el usuario pueda clasificar los elementos que hay disponibles como un ejercicio de reconocimiento de familiares, amigos o hasta lugares y animales. Por ejemplo, dadas unas imágenes que se muestren con diferentes elementos de importancia para el usuario, el propio paciente tiene que indicar el nombre de la persona que reconoce en la fotografía. Y en todo caso, generar una imagen de un recuerdo en concreto con esa misma persona. 


