\chapter*{Abstract}
\renewcommand{\baselinestretch}{1.5}
The world as we know it is advancing more and more and in all the aspects that we know, among these advances we can highlight life expectancy and the use of new technologies. Although at first sight we do not see much relationship between both concepts, they are much closer than we think. The increase in life expectancy is due to improvements in medical and health processes and conditions that have been revolutionary in the last 100 years. Thanks to this, we can live much longer and enjoy life for a few more years. However, there are consequences suffered by people who reach the most advanced years of age, among many, the most important are the deterioration of some sensory abilities, such as sight or hearing; and cognitive abilities such as orientation or memory. The latter is the basis of the main disease that affects older people, Alzheimer's, which directly attacks memory, slowly destroying its capacity until it reaches the most basic functions. This is when the latest technological innovations come in, where we introduce the concept of Artificial Intelligence. Specifically, the generation of images through it. Our objective is to create memories represented by images generated automatically and at the moment, through the words of the person affected by Alzheimer's, in order to build a life book, which will help them through reminiscence therapies that encourage the disease to progress gradually.


\section*{Keywords}

\noindent Artificial intelligence, image generation, prompt, Alzheimer's, reminiscence therapies.



