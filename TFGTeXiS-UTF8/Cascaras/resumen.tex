\chapter*{Resumen}

El mundo tal y como lo conocemos está avanzando cada vez más y en todos los aspectos que conocemos, entre estos avances podemos destacar la esperanza de vida y el uso de las nuevas tecnologías. Aunque en principio no veamos mucha relación entre ambos conceptos, están mucho más unidos de lo que pensamos. El aumento de la esperanza de vida se debe a las mejoras de los procesos y condiciones médicas y sanitarias que han sido de revolución en los últimos 100 años, gracias a ello podemos vivir mucho más tiempo y disfrutar de la vida unos años más. Sin embargo, hay consecuencias que sufren las personas que alcanzan los años más avanzados de edad, entre tantas, las más importantes son el deterioro de algunas capacidades sensoriales, como lo pueden ser la vista o el oído; y capacidades cognitivas como la orientación o la memoria. En esta última es en la que se basa la principal enfermedad que afecta a las personas de mayor edad, el Alzheimer, que ataca directamente a la memoria destruyendo lentamente su capacidad hasta llegar a las funciones más básicas. Aquí es cuando entran las últimas innovaciones tecnológicas realizadas, en dónde introducimos el concepto de Inteligencia Artificial para la realización de este trabajo. En concreto, la generación de imágenes a través de esta. Nuestro objetivo es crear recuerdos representados por imágenes generadas automáticamente y al momento, a través de palabras del propio afectado por el Alzheimer, para poder construir un libro de vida, que le ayudará a través de terapias de reminiscencia que favorecen que la enfermedad avance paulatinamente. 


\section*{Palabras clave}
   
\noindent Inteligencia artificial, generación de imágenes, prompt, Alzheimer, Stable Diffusion, redes neuronales.

   


